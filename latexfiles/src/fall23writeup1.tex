\documentclass[letterpaper,12pt]{article}
\usepackage{fancyhdr}
\usepackage{amsfonts,amsmath,amsthm,amssymb,mathtools,empheq,array}
\usepackage{pgfplots,float,parskip,nameref,graphicx,subfig}
\usepackage[font=small,labelfont=bf]{caption}
\usepackage{pythonhighlight}

\renewcommand{\thefootnote}{\fnsymbol{footnote}}

\graphicspath{{../../.used/figures/}{../assets/}}
\pagestyle{fancy}
\fancyhf{}
\lhead{Fall 2023}
\rhead{Writeup 1}
\rfoot{\thepage}
\title{Computationally Modeling Propagation of a Wave Packet}
\author{Seth Eshraghi}
\date{\today}

\begin{document}
    \maketitle
    \thispagestyle{empty}

    \begin{abstract}
        To calculate expected values of physical observables such as dipole
        moment, velocity, and acceleration, I represented a wave packet with a
        Gaussian envelope as a vector in discrete space and solve a linear
        systems problem to compute propagations at different time steps using
        the implicit propagation scheme. The wave packet is in a box, whose size
        emulates infinte space. Certain consequences of discretizing space and
        time are assessed. Using different parameters, particularly the center
        wave number and width of a potential well, I discovered that
        computational calculations for transmission in a system with a finite
        square potential well might align with the analytically derived
        transmission coefficient.
    \end{abstract}

    \newpage

    \pagenumbering{roman}
    \tableofcontents
    \listoffigures

    \newpage

    \pagenumbering{arabic}
    \addcontentsline{toc}{section}{Introduction}
    \section*{Introduction}

    Computationally studying a physical observable of a single-electron atom may
    require the use of a wave function represented in a
    computer.
    To compute the expected value of any physical quantity, one may make use of
    the probability distribution function (PDF) definition of expected value.
    The square magnitude of a solution to the Schrödoinger equation over
    position at a given time may be used as a PDF if it is normalized. It is
    very difficult to
    determine the state of a wave function at a given time in a certain system
    by analytical methods, so computational modeling of the wave functions is
    used.

    I used the implicit scheme given an initial discretized wave function
    to propagate the wave function through discrete time. The implicit scheme
    and discretization of time and space introduces some limitations, but these
    limitations are predictable. This will allow for easier computation of
    expected values for physical quantities of interest, such as dipole
    acceleration, on an atomic level, which can be rescoped to a macro-level
    system using Maxwell's equations. This process can aid in understanding and
    manipulating certain atomic and molecular processes when combined with
    experimental methods such as utilizing high harmonic generation to create
    attosecond pulses of light.

    The programs implement a basic particle-in-a-box system. The wave packet
    was
    constructed by multiplying a Gaussian function by a plane wave equation, and
    its evolution over time was computed iteratively by the implicit scheme of
    propagation. We investigate a
    system with constant potential energy as well as a potential well. The
    programs are verified by comparing computationally calculated expected values to
    those which were
    analytically calculated as well as assessing the autocorrelation. The results of different
    condidtions are examined graphically, and physical and
    mathematical explanations to those results are provided.

    \newpage

    \addcontentsline{toc}{section}{Theory}
    \section*{Theory}

    Storage and memory on computers are finite, so a wave function or packet and
    the dynamics of it must be represented discretely. Both time and space are
    discretized into a grid with units $\delta t$ and $\delta x$, respectively,
    and we work within one dimension of space. We choose units such that $\hbar$
    equal to one, and we assume that mass is one. Unless stated otherwise,
    change is assumed to be with respect to time.

    \addcontentsline{toc}{subsection}{Implicit Scheme of Propegation}
    \subsection*{Implicit Scheme of Propagation}

    I chose to implicitly propegate a wave function by virtue of computational
    simplicity. We start with the time-dependent Schrödinger equation.

    \[
        \hat{H} \Psi(x, t) = \hat{E} \Psi(x, t)
    \]

    where $\hat{H} = \frac{\hat{p} ^ 2}{2} + V(x)$. We may express the energy
    operator explicitly and then solve differential equation for $\Psi$.

    \begin{align*}
        \hat{H} \Psi(t)
        &= i\frac{\partial}{\partial t} \Psi(t) \\
        \frac{\partial}{\partial t} \Psi(t)
        &= -i\hat{H} \Psi(t) \\
        \Psi(t) &= e^{-i\hat{H} t} \Psi(0) \\
        \Psi(t) &=
        \frac{e^{-i\hat{H} \frac{t}{2}}}{e^{i\hat{H} \frac{t}{2}}} \Psi(0)
    \end{align*}

    We expand both the numerator and the denomenator of the operator into a
    first degree Maclaurin series.

    \[
        \left( 1 + i\hat{H} \frac{t}{2} \right) \Psi(t) \approx
        \left( 1 - i\hat{H} \frac{t}{2} \right) \Psi(0)
    \]

    \pagebreak

    Considering that we are working in discrete space and time, we have

    \begin{equation}
        \left( 1 + iH \frac{\delta t}{2} \right) \vec{\Psi}(\delta t) \approx
        \left( 1 - iH \frac{\delta t}{2} \right) \vec{\Psi}(0)
        \label{impsch}
    \end{equation}

    where $H$ is the matrix representation of the Hamiltonian operator and
    $\vec{\Psi}(t)$ is the column vector representation of $\Psi(t)$ such that
    $\vec{\Psi}(t)_j$ is the value of $\Psi(t)$ at $x = j \cdot \delta x$. The
    wave function at the next time step is computed by solving the linear system
    through forward elimination and backsubstitution.

    \paragraph{Total Energy} Suppose we have a wave fuction representing a plane
    wave, $\psi(x) = e^{ikx}$. Assuming $\hat{H} = \frac{\hat{p}^2}{2}$, we
    have, in discrete space, from the time-independent Schödinger equation

    \[
        -\frac{1}{2}\vec{\psi}_x'' = E \vec{\psi}_x.
    \]

    Now we take the numerical second derivative of $\vec{\psi}_x$ by calculating
    the sum of the Taylor series expansions for $\vec{\psi}_{x - \delta t}$ and
    $\vec{\psi}_{x + \delta t}$ and then solve for $E$.

    \begin{gather*}
        -\frac{\vec{\psi}_{x - \delta t} - 2\vec{\psi}_x + \vec{\psi}_{x +
        \delta t}}{2 \delta t^2} + \mathcal{O}\left[\delta t^2\right]
        = E \vec{\psi}_x
        \\
        -e^{ikx}
        \left(
        \frac{e^{-ik\delta t} + e^{ik\delta t} - 2}{2 \delta t^2}
        \right)
        \approx E e^{ikx}
        \\
        -\frac{2\cos(k)\delta t - 2}{2\delta t^2} \approx E
    \end{gather*}

    \pagebreak

    We examine the behavior of energy as the time step becomes small.

    \begin{align}
        \lim_{\delta t \to 0} E
        &\approx
        \lim_{\delta t \to 0} -\frac{\cos{k\delta t} - 1}{\delta t^2}
        \nonumber
        \\
        &\approx \lim_{\delta t \to 0} \frac{k\sin{k\delta t}}{2\delta t}
        \nonumber
        \\
        &\approx \lim_{\delta t \to 0} \frac{k^2\cos{k\delta t}}{2}
        \nonumber
        \\
        &\approx \lim_{\delta t \to 0}
        \frac{k^2 \left( 1 - \frac{(k\delta t) ^ 2}{2} \right) +
        \mathcal{O}\left[\delta t^4\right]}{2}
        \nonumber
        \\
        E &\approx \frac{k^2}{2}
        \label{eq:EnoV}
    \end{align}

    Note that the energy in discrete space varies with the time step and will
    never be exactly $\frac{k^2}{2}$ as stored in the computer. Additionally,
    the maximum energy of the wave is given by $\frac{2}{\delta x^2}$.

    \paragraph{Error} In our derivation of equation \eqref{impsch}, we
    approximated the exponential operators by truncating the Taylor series
    expansions of them. Note that $e^{-i\hat{H}t} \approx
    \frac{1 - iH\frac{\delta t}{2}}{1 + iH\frac{\delta t}{2}}$. If we expand the
    left side of the approximate equation to its Maclaurin series and multiply
    both sides by $1 + iH\frac{\delta t}{2}$, then we arrive at

    \[
        1 - i\hat{H}\frac{\delta t}{2} - i\hat{H}^3\frac{\delta t^3}{12}
        \approx
        1 - iH\frac{\delta t}{2}
    \]

    so the error for the propagation operator is on the order of $\hat{H}^3
    \delta t^3$. This error will propagate through time.

    \paragraph{Unitary Operator} Because of the numerical second derivative, we
    know that $H$ is a tridiagonal real matrix, so it is Hermitian as an
    operator. In order to use the wave function to calculate expected values,
    the squared magnitude of the wave function must be a PDF over $x$.
    Therefore, the wave function's squared magnitude integrated over all $x$ (or
    in discrete space, the norm) must equal to one for all time, so our operator
    in equation \eqref{impsch} should be unitary.

    \begin{align*}
        \left(
        \frac{1 - iH\frac{\delta t}{2}}{1 + iH\frac{\delta t}{2}}
        \right)^\dag
        \left(
        \frac{1 - iH\frac{\delta t}{2}}{1 + iH\frac{\delta t}{2}}
        \right)
        &=
        \frac{\left(1 + iH^\dag\frac{\delta t}{2}\right)
        \left(1 - iH\frac{\delta t}{2}\right)}
        {\left(1 - iH^\dag\frac{\delta t}{2}\right)
        \left(1 + iH\frac{\delta t}{2}\right)}
        \\
        &=
        \frac{\left(1 + iH\frac{\delta t}{2}\right)
        \left(1 - iH\frac{\delta t}{2}\right)}
        {\left(1 - iH\frac{\delta t}{2}\right)
        \left(1 + iH\frac{\delta t}{2}\right)}
        \\
        &= 1
    \end{align*}
    \vspace{-0.9cm}

    \addcontentsline{toc}{subsection}{Model Boundary Conditions and Parameters}
    \subsection*{Model Boundary Conditions and Parameters}
    We model a one-dimensional particle in a box, where the wave packet is in an
    infinite potential well bounded at $x = 0, R$, where $R$ is large enough to
    emulate an infinite $x$ space for the wave packet. By taking the limit of a
    wave functions value as the potential energy tends toward infinity, we see
    that the magnitude of the wave function is equal to 0 at at the $x$
    boundaries. To reiterate, the grid spacing for time and space is given by
    $\delta t$ and $\delta x$ respectively. Finally, we propagate for a definite
    time. Also note the the parameters for a Gaussian wave packet such as the
    wave number and standard deviation.

    \addcontentsline{toc}{subsection}{Computer Representation of the Hamiltonian
    and Wave Function}
    \subsection*{Computer Representation of the Hamiltonian and Wave Function}

    \paragraph{Wave Function as a Column Vector}
    A wave packet is a superposition of plane waves that have a nonzero range
    in wave number and angular frequency which forms a Gaussian envelope. For a
    given time, it can be created by
    the form

    \vspace{-0.5cm}
    \[
        \Psi(x) = G(x)e^{ikx}
    \]

    where $\Psi$ is a wave packet, $G$ a gaussian function, and $e^{ikx}$ a
    plane wave expression. For storing on a machine, I represented $\Psi(x)$ as

    \vspace{-0.2cm}
    \[
        \vec{\Psi}(t) =
        \begin{bmatrix}
            \Psi(1 \delta x, t) \\
            \Psi(2 \delta x, t) \\
            \vdots \\
            \Psi(N \delta x, t) \\
        \end{bmatrix}
        =
        \begin{bmatrix}
            \Psi_1(t) \\
            \Psi_2(t) \\
            \vdots \\
            \Psi_N(t) \\
        \end{bmatrix}
    \]

    where $N$ is the number of grid spaces on $x$.

    \paragraph{The Propagator Matrix}

    Let us denote $V(j \delta x)$ as $V_j$. Equation \eqref{impsch} can be
    expanded to

    \begin{gather}
        \scalebox{0.7}{$
        \Psi_j(\delta t) + i\frac{\delta t}{2}
        \left(
        -\frac{\Psi_{j-1}(\delta t)-2\Psi_j(\delta t)+\Psi_{j+1}(\delta t)}
        {2 \delta x^2}
        + V_j \Psi_j(\delta t)
        \right)
        $}
        =
        \scalebox{0.7}{$
        \Psi_j(0) - i\frac{\delta t}{2}
        \left(
        -\frac{\Psi_{j-1}(0)-2\Psi_j(0)+\Psi_{j+1}(0)}
        {2 \delta x^2}
        + V_j \Psi_j(0)
        \right)
        $}
        \nonumber
        \\
        \scalebox{0.65}{$
        \begin{bmatrix}
            1 + i\frac{\delta t}{2}\left(V_1 + \frac{1}{\delta t^2}\right)
            &
            -i\frac{\delta t}{(2 \delta x)^2} & 0 & 0 & 0 & \cdots & 0 \\
            -i\frac{\delta t}{(2 \delta x)^2}
            &
            1 + i\frac{\delta t}{2}\left(V_2 + \frac{1}{\delta t^2}\right)
            & -i\frac{\delta t}{(2 \delta x)^2} & 0 & 0 & \cdots & 0 \\
            0 & \ddots & \ddots & \ddots & 0 & \cdots & 0 \\
            \vdots & & & & & & \vdots \\
            0 & \cdots & \cdots & 0 &
            -i\frac{\delta t}{(2 \delta x)^2}
            &
            1 + i\frac{\delta t}{2}\left(V_2 + \frac{1}{\delta t^2}\right)
            & -i\frac{\delta t}{(2 \delta x)^2}
        \end{bmatrix}
        \begin{bmatrix}
            \Psi_1(\delta t) \\
            \Psi_2(\delta t) \\
            \vdots \\
            \\
            \Psi_N(\delta t)
        \end{bmatrix}
        =
        \vec{\phi}_j(0)
        $}
        \label{prop}
    \end{gather}

    where the matrix in equation \eqref{prop} is the propagator matrix. It is
    straightforward to guess the algorithm that is used to repeatedly propagate
    the wave function column vector through time.

    \addcontentsline{toc}{subsection}{Finite Square Potential Well}
    \subsection*{Finite Square Potential Well}
    A potential well is a region where the wave packet experiences a lower
    amount of potential energy. A well can be modeled by defining the potential
    energy function $V(x)$ to include a region with a lower value.

    In computer implementation, the well must be continuous due to the Gibbs
    phenomenon. A smooth square well that does not allow for the occurance of
    Gibbs
    phenomenon can be achieved by using a higher power Gaussian function. In my
    modeling, I set the width of the well to be smaller than the width of the
    wave packet, which will be discussed in more detail within the
    \nameref{sec:Results} section.

    \addcontentsline{toc}{subsubsection}{Transmission Coefficient}
    \subsubsection*{Transmission Coefficient}
    As the wave propagates through the finite potential well, some of it is
    reflected backward, and some of it is transmitted through to the other side
    of the well.

    Suppose a potential well has a width of $2a$ and is centered at $x = 0$,
    where the potential $V(x)$ is equal to $-V_0$ for $-a \leq x \leq a$ and $0$
    for $x > a$ and $x < -a$. We label the region corresponding to $x < -a$ as
    region I, to $-a < x < a$ as region II, to $x > a$ as region III. During our
    meetings we solved for the one-dimentional time-independent Schrödinger
    equation for when $E > V(x)$ and $E < V(x)$, obtaining

    \begin{align*}
        \Psi_\mathrm{I}(x) &= Ae^{ikx} + Be^{-ikx} \\
        \Psi_{\mathrm{II}}(x) &= C\sin(lx) + D\cos(lx) \\
        \Psi_{\mathrm{III}}(x) &= Fe^{ikx}
    \end{align*}

    where $l = \sqrt{2(E + V_0)}$, $k = \sqrt{2E}$, and the subscript of the
    denotation of each function indicates over which region the function is
    defined. The requirements to stitch the three functions into one function
    which is defined for all $x$
    follows the properties of a valid solution to the Schrödinger equation:

    We compute the transmission coefficient ($T$) by definition ($T^{-1} =
    \frac{\left| A \right| ^{2}}{\left| F \right| ^{2}}$), substituting $F$ with
    equivalent expression in terms of the energy of the wave ($E$) and the
    depth of the potential well. We choose units such that $\hbar = 1$ and
    asuume $m = 1$.

    The requirements to stitch the three wave functions into one follows the
    properties of a valid solution to the Schrödinger equation:

    \begin{gather}
        Ae^{-ika} + Be^{ika} = -C\sin(la) + D\cos(la) \label{stitch1} \\
        C\sin(la) + D\cos(la) = Fe^{ika} \label{stitch2} \\
        ik(Ae^{-ika} - Be^{ika}) = l(C\cos(la) + D\sin(la)) \label{stitch3} \\
        l(C\cos(la) - D\sin(la)) = ikFe^{ika} \label{stitch4}
    \end{gather}

    Now we solve for $F$ in terms of $l$, $k$, $a$, and $A$. From
    equation \eqref{stitch2}, we solve

    \begin{equation}
        C = \frac{Fe^{ika} - D\cos(la)}{\sin(la)} \label{CinD}
    \end{equation}

    and substitute into equation \eqref{stitch4} to solve for

    \begin{gather*}
        Fe^{ika}
        \left(
        \frac{ik}{l}\sin(la) - \cos(la)) = -D(\cos^2(la) + \sin^2(la)
        \right)
        \\
        D = Fe^{ika}
        \left(
        \cos(la) - \frac{ik}{l}\sin(la)
        \right)
    \end{gather*}

    Then we substitute $D$ in terms of $F$ into equation \eqref{CinD} and
    simplify. Note $1 - \cos^2(la) = 1 - \frac{1}{2}(1 + \cos(2la)) = 1 -
    \frac{1}{2}(1 + 1 - 2\sin^2(la)) = \sin^2(la)$.

    \[
        \scalebox{1}{$
        C
        = \frac{Fe^{ika}(1 - \cos^2(la) +
        \frac{ik}{l}\cos(la)\sin(la))}{\sin(la)}
        = Fe^{ika}
        \left(
        \frac{1 - \cos^2(la)}{\sin(la)} + \frac{ik}{l}\cos(la)
        \right)
        $}
    \]

    \[
        C = Fe^{ika}
        \left(
        \sin(la) + \frac{ik}{l}\cos(la)
        \right)
    \]

    Note $\cos^2(la) - \sin^2(la) = \cos(2la)$ and $2\cos(la)\sin(la) =
    \sin(2la)$. Substituting $D$ and $C$ in terms of $F$ into equation
    \eqref{stitch1}, we get

    \begin{gather}
        \scalebox{0.9}{$
        Ae^{-ika}
        = Fe^{ika}
        \left(
        -\sin^2(la) - \frac{ik}{l}\cos(la)\sin(la) + \cos^2(la)
        - \frac{ik}{l}\sin(la)\cos(la)
        \right)
        - Be^{ika}
        \nonumber
        $}
        \\
        = Fe^{ika}
        \left(
        \cos(2la) - \frac{ik}{l}\sin(2la)
        \right)
        - Be^{ika}. \label{AinFB}
    \end{gather}

    Now we substitute this result and $D$ and $C$ in terms of $F$ into equation
    \eqref{stitch3} and solve for $B$.

    \begin{align*}
        \scalebox{0.8}{$
        Fe^{ika}
        \left(
        -\frac{ik}{l}\sin(2la) + \cos(2la)
        \right)
        - 2Be^{ika}
        $}
        &=
        \scalebox{0.8}{$
        \frac{l}{ik}Fe^{ika}
        \left(
        2\cos(la)\sin(la) + \frac{ik}{l}\cos^2(la) - \frac{ik}{l}\sin^2(la)
        \right)
        $}
        \\
        -2B &= F\sin(2la)
        \left(
        \frac{l}{ik} + \frac{ik}{l}
        \right)
        \\
        B &= i\frac{\sin(2la)}{2kl}(l^2 - k^2)F
    \end{align*}

    To solve for $F$ in terms of $A$ we substitute $B$, $C$, and $D$ into
    equation \eqref{AinFB}.

    \begin{gather*}
        Ae^{-ika}
        = Fe^{ika}
        \left(
        \cos(2la) - \frac{ik}{l}\sin(2la)
        - i\frac{\sin(2la)}{2kl}(l^2 - k^2)
        \right)
        \\
        Ae^{-2ika}
        = Fe^{ika}
        \left(
        \cos(2la) -
        \left(
        \frac{ik}{l} + \frac{i(l^2 - k^2)}{2lk}
        \right)
        \sin(2la)
        \right)
        \\
        F = \frac{e^{-2ika}A}{\cos(2la) - i
        \left(
        \frac{k^2 + l^2}{2kl}
        \right)
        \sin(2la)}
    \end{gather*}

    Finally, we compute $T^{-1}$ and rewrite it in terms of $E$ and $V_0$.

    \begin{gather*}
        T^{-1} = \frac{\left| A \right| ^{2}}{\left| F \right| ^{2}}
        = \frac{\left| A \right| ^{2}}{
            \frac{\left| e^{-2ika} \right| ^2 \left| A \right | ^2}{
                    \left|
                    \cos^2(2la) + i \left( \frac{k^2 + l^2}{2kl} \right)
                    \sin(2la)
                    \right| ^2
                }
            }
        = \cos^2(2la) + \left( \frac{(k^2 + l^2)^2}{4k^2l^2} \right) \sin^2(2la)
        \\
        = 1 - \sin^2(2la) + \left( \frac{(k^2 + l^2)^2}{4k^2l^2}
        \right) \sin^2(2la)
        = 1 + \left( \frac{k^4 - 2k^2l^2 + l^4}{4k^2l^2} \right) \sin^2(2la)
        \\
        = 1 + \left( \frac{4E^2 - 8E(E + V_0) + 4(E^2 + 2EV_0 + V_0^2)}
        {16E(E + V_0)} \right) \sin^2(2a\sqrt{2(E + V_0)}
    \end{gather*}

    \begin{center}
        \boxed{
            T = \frac{1}{1 + \left( \frac{V_0}{4E(E + V_0)} \right)
            \sin^2(2a\sqrt{2(E + V_0)}}
        }
    \end{center}

    Note that T oscillates with changes in the width of the well and the energy
    of the wave.

    \addcontentsline{toc}{subsection}{Expected Values over Time}
    \subsection*{Expected Values over Time}
    \label{sec:expected}

    In the \nameref{subsec:verify} subsection, I will be numerically calculating
    the expected values of certain physical observables of the wave packet
    through time to gain confidence in correctness of my programs. In this
    section, we will study the theoretic evolution of those values, and the
    potential energy functions of interest are determined by systems which were
    modeled computationally.

    Both systems are closed, so energy is conserved and the value of the
    Hamiltonian does not change.

    \addcontentsline{toc}{subsubsection}{Constant Potential Energy}
    \subsubsection*{Constant Potential Energy}

    From \eqref{eq:EnoV}, we can deduce that $p \approx k$. As the wave packet
    approaches a wall, the expected momentum will decrease as some of the wave
    packet will have been reflected backward, while the rest of the wave packet
    is still moving forward. The expected momentum will continue to decrease
    until the wave packet has been fully reflected, at which point $p \approx
    -k$.

    Recall that a wavepacket is a superposition of
    eigenstates, and that it has multiple, distinct $k$ values. One may
    calculate the distribution of the $k$ values from a Fourier transform, and
    would notice that it corresponds to the Gaussian envelope of the wave
    packet. The components of the wave
    packet with larger corresponding wave numbers will have larger momentum than
    those with smaller corresponding wave numbers, so at every reflection,
    change in momentum after the inflection point is less drastic.

    The expected momentum is proportional to the change in the expected
    position (in our case, it exactly describes the motion of the wave packet
    because we assume that mass is one), so the expected $x$ would increase to a
    certain point linearly, then decrease linearly to a certain point, and so
    on. Because of the range of wave numbers discussed above, the
    change in $x$ of the wave packet becomes larger, and the maxiumum expected
    $x$ for each period is less than that for the previous period.

    \addcontentsline{toc}{subsubsection}{Finite Square Potential Well}
    \subsubsection*{Finite Square Potential Well}

    Just as in a system with a constant potential function, the momuntum of the
    wave packet is initially approximately $k$ in a system with a finite
    potential well. However, once the wave packet passes the region with lower
    potential, part of it is transmitted as a wave packet whose momentum has the
    same sign as the momentum of the original wave packet, and the rest
    reflected as another wave packet whose momentum has the opposite sign as the
    momentum of the original wave packet. Otherwise, it is
    negative. The larger the probability of the particle that is localized by
    the wave packet being past the potential well after transmission, the larger
    the expected momentum will be. If more of the wave packet is
    transmitted, the momentum between the time of finishing transmission and
    reflection off
    the boundaries of the infinite square well is positive.

    Of course, the behavior of the momentum at reflection off the boundaries of
    the box is consistent with that within a box with constant potential, and
    the consequences of the wave packet having a range of wave numbers holds in
    this system. Again, the change of the expected position is proportional to
    the momentum.

    \paragraph{Effects of Changing Well Width and Energy}  We know that
    the behavior of expected momentum and position in a finite square well are
    contingent on the transmission of the initial wave packet(s) and that the
    transmission changes with the width of the well and the energy of the wave
    packet. Thus the behavior of the those expected values change with the width
    of the well and the energy of the wave packet. Changing the value of either
    with the other held constant oscillates the value of the transmission
    coefficient, and the changes with respect to the width of the well and
    the energy of the wave packet in the evolution of expected
    momentum and expected position follow from the descriptions detailed above.

    \newpage

    \addcontentsline{toc}{section}{Results}
    \section*{Results}
    \label{sec:Results}

    All the necessary concepts in the theory for the model have been covered,
    so we are free to fully discuss the computational portion.

    \addcontentsline{toc}{subsection}{Verifying Implementation}
    \subsection*{Verifying Implementation}
    \label{subsec:verify}

    Before we invesigate the results of the computational modeling, we gain
    confidence in the correctness of the programs by checking values over time.
    We also use autocorrelation to further support the correctness of the
    propagation.

    \addcontentsline{toc}{subsubsection}{Examining Expected Values}
    \subsubsection*{Examining Expected Values}

    First, we check if my implemetned operator is unitary.
    For all graphs, the $x$ axis is displaying time unless specified otherwise.
    The final time is not changed.

    \begin{figure}[H]
        \centering
        \includegraphics[width=0.6\textwidth]{norm.png}
        \caption[Squared Magnitude of the Wave Function over Time]{The
        \texttt{norm} axis indicates the sum of all the elements of
        $\vec{\Psi}(t)$.}
    \end{figure}

    The norm is one through all our time steps, so the implemented operator is
    unitary, as we calculated.
    Now we examine the expected total energy, momentum, and position for
    different $k$ values for a system with a constant potential function and a
    potential well.

    \pagebreak

    \begin{center}
        \textit{Constant Potential}
    \end{center}

    We check to see if $\langle H \rangle \approx \frac{k^2}{2}$ and $\left|
    \frac{d}{dt} \langle x \rangle \right| = \left| \langle p \rangle \right|
    \approx k$, and that they behave roughly as explained in the
    \nameref{sec:expected} section.

    For $k = 0.59$, I computed with the programs
    the following values.

    \begin{figure}[H]
        \centering
        \begin{tabular}{cc}
            \includegraphics[width=0.4\textwidth]{Hc0.59.png}
            &
            \includegraphics[width=0.4\textwidth]{pc0.59.png}
            \\
            (a) & (b)
            \\
            \multicolumn{2}{c}
            {\includegraphics[width=0.4\textwidth]{xc0.59.png}}
            \\
            \multicolumn{2}{c}{(c)}
        \end{tabular}
        \caption[Expected Values over Time for $k = 0.59$]{The three plots
        display the expected values over time for $k =
        0.59$. The $y$-axis in plot (a) indicates the total energy of
        the system. The $y$-axis in plots (b) and (c) indicate the
        expected momentum and expected position of the wave packet,
        respectively.}
        \label{fig:c0.59}
    \end{figure}

    The value $\frac{d}{dt} \langle x \rangle$ is equal to the slope in plot (c)
    in Figure
    \ref{fig:c0.59}, which can be computed using the coorindates $(x, t)
    = (0, 80), (81, 127.34)$ from the datafile: $\frac{d}{dt} \langle x \rangle
    \approx 0.58 \approx k$. The computational expectation values for the total
    energy of the
    system and momentum of the wave packet are consistent with the theory.

    \pagebreak

    Now for $k = 1$, the programs output the following data.

    \begin{figure}[H]
        \centering
        \begin{tabular}{cc}
            \includegraphics[width=0.4\textwidth]{Hc1.png}
            &
            \includegraphics[width=0.4\textwidth]{pc1.png}
            \\
            (a) & (b)
            \\
            \multicolumn{2}{c}{\includegraphics[width=0.4\textwidth]{xc1.png}}
            \\
            \multicolumn{2}{c}{(c)}
        \end{tabular}
        \caption[Expected Values over Time for $k = 1$]{The three plots display
        the expected values over time for $k =
        1$. The $y$-axis in plot (a) indicates the total energy of
        the system. The $y$-axis in plots (b) and (c) indicate the
        expected momentum and expected position of the wave packet,
        respectively.}
        \label{fig:c1}
    \end{figure}

    The value $\frac{d}{dt} \langle x \rangle$ is equal to the slope in plot (c)
    in Figure
    \ref{fig:c1}, which can be computed using the coorindates $(x, t)
    = (0, 80), (81, 156.43)$ from the datafile: $\frac{d}{dt} \langle x \rangle
    \approx 0.94 \approx k$. The wave packet moves over more distance in the
    same amount of time for a larger $k$. The computational expectation values
    are consistent with the theory.

    \pagebreak

    \begin{center}
        \textit{Finite Potential Square Well}
    \end{center}

    I used the following potential well.

    \begin{figure}[H]
        \centering
        \includegraphics[width=0.6\textwidth]{smallwell.png}
        \caption[Small Potential Well]{The $y$-axis represents the potential
        energy, and the $x$-axis represents the position within the box.}
        \label{fig:smallwell}
    \end{figure}

    We check to see if $\langle H \rangle \approx \frac{k^2}{2}$ and if
    initially $\left|
    \frac{d}{dt} \langle x \rangle \right| = \left| \langle p \rangle \right|
    \approx k$, and that they behave roughly as explained in the
    \nameref{sec:expected} section. The width of the well used was $0.01R$, and
    the full width at half maximum (FWHM) of the Gaussian scalar function of the
    initial wave packet used is $20\sqrt{2\log(2)}$ in $x$ space.

    Figure \ref{fig:w} on the next page shows the numerical computations of the
    expected value of each of the physical quantities for $k = 0.59, 1$, as
    calculated by the programs.

    \begin{figure}[H]
        \centering
        \begin{tabular}{ccc}
            \includegraphics[width=0.3\textwidth]{Hw0.59.png}
            &
            \includegraphics[width=0.3\textwidth]{pw0.59.png}
            &
            \includegraphics[width=0.3\textwidth]{xw0.59.png}
            \\
            (a) & (b) & (c)
            \\
            \includegraphics[width=0.3\textwidth]{Hw1.png}
            &
            \includegraphics[width=0.3\textwidth]{pw1.png}
            &
            \includegraphics[width=0.3\textwidth]{xw1.png}
            \\
            (d) & (e) & (f)
        \end{tabular}
        \caption[Expected Values for a Large Well for $k = 0.59, 1$]{The top
        three plots display the expected values over time for
        $k = 0.59$, and the bottom three for $k = 1$. The $y$-axis in plots (a)
        and (d) indicates the total energy of the system. The $y$-axis in plots
        (b) and (e) indicates the expected momentum, and the $y$-axis in
        plots (c) and (f) indicates the expected position of the wave packet.}
        \label{fig:w}
    \end{figure}

    Although we will not compute the initial slope of plots (c) and (f), we may
    notice that the predicted change in the expected position is represented in
    the graphs, so the computational expected values are consistent with the
    theory. The discrepancies between the graphs for $k = 0.59$ and $k = 1$ will
    be discussed in further detail in the \nameref{subsec:num} subsection.

    \addcontentsline{toc}{subsubsection}{Autocorrelation for $\left| \Psi
    \right| ^ 2$ through Negative Time}
    \subsubsection*{Autocorrelation for $\left| \Psi \right| ^ 2$ through
    Negative Time}

    We assess correctness with autocorrelation by checking if it is
    approximately equal to one.
    The following is the schema for measuring autocorrelation, where
    $n \in \mathbb{Z}$.

    \begin{align*}
        \Psi(n \delta t) &=
        \left(
        \frac{1 - iH\frac{\delta t}{2}}{1 + iH\frac{\delta t}{2}}
        \right)^n
        \Psi(0)
        \\
        {\tilde{\Psi}}_n(0) &=
        \left(
        \frac{1 - iH\frac{-\delta t}{2}}{1 + iH\frac{-\delta t}{2}}
        \right)^n
        \Psi(n \delta t)
        \\
        \mathrm{autocorrelation} &=
        \int_0^R \Psi(0)^* \tilde{\Psi}(0) dx
    \end{align*}

    To implement this measure using the implicit scheme of propagation, I
    propagated forward in time by $n$ time-steps, and backward in time by the
    same number of time steps. The integral was calculated by summing the
    product of each element of $\vec{\Psi}(0)$ and the complex conjugate of its
    corresponding element of $\vec{\tilde{\Psi}}(0)$.

    \begin{figure}[H]
        \centering
        \includegraphics[width=0.7\textwidth]{autocorr.out.png}
        \caption[Autocorrelation output]{The output for the autocorrelation for
        $\delta t = 0.9$ and the $n = 2222$.}
    \end{figure}

    The autocorrelation is exactly equal to one because the error in forward
    propagation is equal to the negative of the backward negative operator
    because a factor of the propagator error, $\delta t^3$, is odd with respect
    to $\delta t$.

    \addcontentsline{toc}{subsection}{Numerical Calculations of the Transmission
    Coefficient}
    \subsection*{Numerical Calculations of the Transmission Coefficient}
    \label{subsec:num}

    I numerically calculated the transmission coefficient by

    \[
        \sum_{j = A}^N \vec{\Psi}_j(t_f)
    \]

    where $A$ is the number of spacial steps that corresponds to the approximate
    edge of the well that is on the side opposite of the starting position of
    the wave packet, $N$ is again the total number of spacial steps. This
    summation is computed at $t_f$, corresponding to a time after the full
    transmission through the well has occured.

    \begin{figure}[H]
        \centering
        \begin{tabular}{cc}
            \includegraphics[width=0.4\textwidth]{translarge.png}
            &
            \includegraphics[width=0.4\textwidth]{largewell.png}
            \\
            (a) & (b)
        \end{tabular}
        \caption[Large Well Transmission]{The $x$-axis represents $k$. The
        transmission coefficient as calculated analytically is plotted in purple
        and as computed numerically is plotted in turquoise. The length of each
        side of the error bar is the half width of the wave packet in $k$ space,
        calculated by $\frac{2}{\sqrt{2\sigma^2}}$ where $\sigma$ is the standard
        deviation parameter for the Gaussian envelope.}
        \label{translarge}
    \end{figure}

    We see that the numerical computation does not match the analytical solution.
    If we recall that the analytical solution is for an eigenstate,
    and that a wave packet is superposition of many eigenstates, it is then
    reasonable to guess that the plot of the numerical solution appears as
    sort of an $S$-curve because the transmission of all plane waves
    composing the wave packet, each of which corresponds to its own $k$ value,
    are summed, and that the center of the $k$ distribution is larger than one
    oscillation in the analytical solution. We must reduce the width of the wave
    packet in $k$-space in order for the numerical plot to resemble the
    analytical plot. This corresponds to a decrease in the width of the wave
    packet in $x$-space, which at this point we may reduce the width of the well
    to maintain an approximately infinitely large box\footnote[1]{The relation
    between the well width in $x$-space and $k$-space is calculated by Fourier\\
    transform and FWHM of the wave packet.}. The graph (a) in Figure
    \ref{translarge} uses a well width of $0.1R$, seen in graph (b). The
    following figure uses a well width of $0.01R$, seen in Figure
    \ref{fig:smallwell}.

    \begin{figure}[H]
        \centering
        \includegraphics[width=0.6\textwidth]{transsmall.png}
        \caption[Small Well Transmission]{The plots are the same as in Figure
        \ref{translarge}.}
    \end{figure}

    We see now that the error bar is smaller than one oscillation in the
    analytical plot, and that both functions match with an offset of some phase
    constant.

    Revisiting the graphs for the expected values over time for a system with a
    finite potential well (Figure \ref{fig:w}), we noted the differences in the
    expected values over $k = 0.59$ and $k = 1$. If we compare those to the
    graphs in Figures \ref{fig:c0.59} and \ref{fig:c1}, we see that the expected
    values behave more similarly between a constant potential and a potential
    well for $k = 1$ than $k = 0.59$. This is because there is a larger
    transmission at $k = 0.59$, which is supported by the analytical plot.

    \addcontentsline{toc}{subsubsection}{Extended Verification}
    \subsubsection*{Extended Verification}

    We extend our verification process by examining the wave packet and
    transmission over time over $k$. We notice a reflection and transmission in
    Figure \ref{fig:wellanim}, and an oscillation in transmission (equal to the
    plateau since the initial wave packet is normalized) through $k$ in Figure
    \ref{fig:Fanim}.

    \begin{figure}[H]
        \centering
        \begin{tabular}{cc}
            \includegraphics[width=0.4\textwidth]{well1.png}
            &
            \includegraphics[width=0.4\textwidth]{well252.png}
            \\
            \includegraphics[width=0.4\textwidth]{well400.png}
            &
            \includegraphics[width=0.4\textwidth]{well612.png}
        \end{tabular}
        \caption[Well Propagation]{The squared magnitude of the wave packet at
        four different time steps}
        \label{fig:wellanim}
    \end{figure}

    \begin{figure}[H]
        \centering
        \begin{tabular}{ccc}
            \includegraphics[width=0.3\textwidth]{F8.png}
            &
            \includegraphics[width=0.3\textwidth]{F20.png}
            &
            \includegraphics[width=0.3\textwidth]{F50.png}
        \end{tabular}
        \begin{tabular}{cc}
            \includegraphics[width=0.3\textwidth]{F55.png}
            &
            \includegraphics[width=0.3\textwidth]{F65.png}
            % \multicolumn{2}{c}{\includegraphics[width=0.4\textwidth]{F65.png}}
        \end{tabular}
        \caption[Transmission Probabilities]{The probability of the localized
        particle being transmitted ($F$) over time for increasing $k$}
        \label{fig:Fanim}
    \end{figure}
\end{document}
