\addcontentsline{toc}{section}{Introduction}
\section*{Introduction}

Computationally studying a physical observable of a single-electron atom may
require the use of a wave function represented in a computer.  To compute
the expected value of any physical quantity, one may make use of the
probability distribution function (PDF) definition of expected value. The
square magnitude of a solution to the Schrödoinger equation over position at
a given time may be used as a PDF if it is normalized. It is very difficult
to determine the state of a wave function at a given time in some systems by
analytical methods, so computational modeling of the wave functions is used.

I used the implicit scheme given an initial discretized wave function to
propagate the wave function through discrete time. The implicit scheme and
discretization of time and space introduces some limitations, but these
limitations are predictable. This will allow for easier computation of
expected values for physical quantities of interest, such as dipole
acceleration, on an atomic level, which can be rescoped to a macro-level
system using Maxwell's equations. This process can aid in understanding and
manipulating certain atomic and molecular processes when combined with
experimental methods such as utilizing high harmonic generation to create
attosecond pulses of light.

The programs implement a basic particle-in-a-box system. The wave packet was
constructed by multiplying a Gaussian function by a plane wave equation, and
its evolution was computed iteratively by the implicit scheme of
propagation. We investigate a system with constant potential energy as well
as a potential well. The programs are verified by comparing computationally
calculated expected values to those which were analytically calculated as
well as assessing the autocorrelation. The results of different condidtions
are examined graphically, and physical and mathematical explanations to
those results are provided.
