\addcontentsline{toc}{section}{Theory}
\section*{Theory}

Storage and memory on computers are finite, so a wave function or packet and
the dynamics of it must be represented discretely. Both time and space are
discretized into a grid with units $\delta t$ and $\delta x$, respectively,
and we work within one dimension of space. We choose units such that $\hbar$
is equal to one, and we assume that mass is one. Unless stated otherwise,
change is assumed to be with respect to time.

\addcontentsline{toc}{subsection}{Implicit Scheme of Propegation}
\subsection*{Implicit Scheme of Propagation}

I chose to implicitly propegate a wave function by virtue of computational
simplicity. We start with the time-dependent Schrödinger equation.

\[
    \hat{H} \Psi(x, t) = \hat{E} \Psi(x, t)
\]

where $\hat{H} = \frac{\hat{p} ^ 2}{2} + V(x)$. We may express the energy
operator explicitly and then solve differential equation for $\Psi$.

\begin{align*}
    \hat{H} \Psi(t)
    &= i\frac{\partial}{\partial t} \Psi(t) \\
    \frac{\partial}{\partial t} \Psi(t)
    &= -i\hat{H} \Psi(t) \\
    \Psi(t) &= e^{-i\hat{H} t} \Psi(0) \\
    \Psi(t) &=
    \frac{e^{-i\hat{H} \frac{t}{2}}}{e^{i\hat{H} \frac{t}{2}}} \Psi(0)
\end{align*}

We expand both the numerator and the denomenator of the operator into a
first degree Maclaurin series.

\[
    \left( 1 + i\hat{H} \frac{t}{2} \right) \Psi(t) \approx
    \left( 1 - i\hat{H} \frac{t}{2} \right) \Psi(0)
\]

\pagebreak

Considering that we are working in discrete space and time, we have

\begin{equation}
    \left( 1 + iH \frac{\delta t}{2} \right) \vec{\Psi}(\delta t) \approx
    \left( 1 - iH \frac{\delta t}{2} \right) \vec{\Psi}(0)
    \label{eq:impsch}
\end{equation}

where $H$ is the matrix representation of the Hamiltonian operator and
$\vec{\Psi}(t)$ is the column vector representation of $\Psi(t)$ such that
$\vec{\Psi}(t)_j$ is the value of $\Psi(t)$ at $x = j \cdot \delta x$. The
wave function at the next time step is computed by solving the linear system
through forward elimination and backsubstitution.

\paragraph{Total Energy} Suppose we have a wave fuction representing a plane
wave, $\psi(x) = e^{ikx}$. Assuming $\hat{H} = \frac{\hat{p}^2}{2}$, we have,
in discrete space, from the time-independent Schödinger equation

\[
    -\frac{1}{2}\vec{\psi}_x'' = E \vec{\psi}_x.
\]

Now we take the numerical second derivative of $\vec{\psi}_x$ by calculating
the sum of the Taylor series expansions for $\vec{\psi}_{x - \delta t}$ and
$\vec{\psi}_{x + \delta t}$ and then solve for $E$.

\begin{gather*}
    -\frac{\vec{\psi}_{x - \delta t} - 2\vec{\psi}_x + \vec{\psi}_{x +
    \delta t}}{2 \delta t^2} + \mathcal{O}\left[\delta t^2\right]
    = E \vec{\psi}_x
    \\
    -e^{ikx}
    \left(
    \frac{e^{-ik\delta t} + e^{ik\delta t} - 2}{2 \delta t^2}
    \right)
    \approx E e^{ikx}
    \\
    -\frac{2\cos(k)\delta t - 2}{2\delta t^2} \approx E
\end{gather*}

\pagebreak

We examine the behavior of energy as the time step becomes small.

\begin{align}
    \lim_{\delta t \to 0} E
    &\approx
    \lim_{\delta t \to 0} -\frac{\cos{k\delta t} - 1}{\delta t^2}
    \nonumber
    \\
    &\approx \lim_{\delta t \to 0} \frac{k\sin{k\delta t}}{2\delta t}
    \nonumber
    \\
    &\approx \lim_{\delta t \to 0} \frac{k^2\cos{k\delta t}}{2}
    \nonumber
    \\
    &\approx \lim_{\delta t \to 0}
    \frac{k^2 \left( 1 - \frac{(k\delta t) ^ 2}{2} \right) +
    \mathcal{O}\left[\delta t^4\right]}{2}
    \nonumber
    \\
    E &\approx \frac{k^2}{2}
    \label{eq:EnoV}
\end{align}

Note that the energy in discrete space varies with the time step and will
never be exactly $\frac{k^2}{2}$ as stored in the computer. Additionally,
the maximum energy of the wave is given by $\frac{2}{\delta x^2}$.

\paragraph{Error} In our derivation of equation \eqref{eq:impsch}, we
approximated the exponential operators by truncating the Taylor series
expansions of them. Note that $e^{-i\hat{H}t} \approx
\frac{1 - iH\frac{\delta t}{2}}{1 + iH\frac{\delta t}{2}}$. If we expand the
left side of the approximate equation to its Maclaurin series and multiply
both sides by $1 + iH\frac{\delta t}{2}$, then we arrive at

\[
    1 - i\hat{H}\frac{\delta t}{2} - i\hat{H}^3\frac{\delta t^3}{12}
    \approx
    1 - iH\frac{\delta t}{2}
\]

so the error for the propagation operator is on the order of $\hat{H}^3
\delta t^3$. This error will propagate through time.

\paragraph{Unitary Operator} Because of the numerical second derivative, we
know that $H$ is a tridiagonal real matrix, so it is Hermitian as an
operator. In order to use the wave function to calculate expected values,
the square magnitude of the wave function must be a PDF over $x$.
Therefore, the wave function's square magnitude integrated over all $x$ (or
in discrete space, the norm) must equal to one for all time, so our operator
in equation \eqref{eq:impsch} should be unitary.

\begin{align*}
    \left(
    \frac{1 - iH\frac{\delta t}{2}}{1 + iH\frac{\delta t}{2}}
    \right)^\dag
    \left(
    \frac{1 - iH\frac{\delta t}{2}}{1 + iH\frac{\delta t}{2}}
    \right)
    &=
    \frac{\left(1 + iH^\dag\frac{\delta t}{2}\right)
    \left(1 - iH\frac{\delta t}{2}\right)}
    {\left(1 - iH^\dag\frac{\delta t}{2}\right)
    \left(1 + iH\frac{\delta t}{2}\right)}
    \\
    &=
    \frac{\left(1 + iH\frac{\delta t}{2}\right)
    \left(1 - iH\frac{\delta t}{2}\right)}
    {\left(1 - iH\frac{\delta t}{2}\right)
    \left(1 + iH\frac{\delta t}{2}\right)}
    \\
    &= 1
\end{align*}
\vspace{-0.9cm}

\addcontentsline{toc}{subsection}{Model Boundary Conditions and Parameters}
\subsection*{Model Boundary Conditions and Parameters}
We model a one-dimensional particle in a box, where the wave packet is in an
infinite potential well bounded at $x = 0, R$, where $R$ is large enough to
emulate an infinite $x$ space for the wave packet. As the potential energy
tends toward infinity, we see that the magnitude of the wave function is
equal to 0 at at the $x$ boundaries. To reiterate, the grid spacing for time
and space is given by $\delta t$ and $\delta x$ respectively. Finally, we
propagate for a definite time. Also note the the parameters for a Gaussian
wave packet such as the wave number and standard deviation.

\addcontentsline{toc}{subsection}{Computer Representation of the Hamiltonian
and Wave Function}
\subsection*{Computer Representation of the Hamiltonian and Wave Function}

\paragraph{Wave Function as a Column Vector}
A wave packet is a superposition of plane waves that have a nonzero range
in wave number and angular frequency which forms a Gaussian envelope. For a
given time, it can be created by the form

\vspace{-0.5cm}
\[
    \Psi(x) = G(x)e^{ikx}
\]

where $\Psi$ is a wave packet, $G$ a gaussian function, and $e^{ikx}$ a
plane wave expression. For storing on a machine, I represented $\Psi(x)$ as

\vspace{-0.2cm}
\[
    \vec{\Psi}(t) =
    \begin{bmatrix}
        \Psi(1 \delta x, t) \\
        \Psi(2 \delta x, t) \\
        \vdots \\
        \Psi(N \delta x, t) \\
    \end{bmatrix}
    =
    \begin{bmatrix}
        \Psi_1(t) \\
        \Psi_2(t) \\
        \vdots \\
        \Psi_N(t) \\
    \end{bmatrix}
\]

where $N$ is the number of grid spaces on $x$.

\paragraph{The Propagator Matrix}

Let us denote $V(j \delta x)$ as $V_j$. Equation \eqref{eq:impsch} can be
expanded to

\begin{gather}
    \scalebox{0.7}{$
    \Psi_j(\delta t) + i\frac{\delta t}{2}
    \left(
    -\frac{\Psi_{j-1}(\delta t)-2\Psi_j(\delta t)+\Psi_{j+1}(\delta t)}
    {2 \delta x^2}
    + V_j \Psi_j(\delta t)
    \right)
    $}
    =
    \scalebox{0.7}{$
    \Psi_j(0) - i\frac{\delta t}{2}
    \left(
    -\frac{\Psi_{j-1}(0)-2\Psi_j(0)+\Psi_{j+1}(0)}
    {2 \delta x^2}
    + V_j \Psi_j(0)
    \right)
    $}
    \nonumber
    \\
    \scalebox{0.65}{$
    \begin{bmatrix}
        1 + i\frac{\delta t}{2}\left(V_1 + \frac{1}{\delta t^2}\right)
        &
        -i\frac{\delta t}{(2 \delta x)^2} & 0 & 0 & 0 & \cdots & 0 \\
        -i\frac{\delta t}{(2 \delta x)^2}
        &
        1 + i\frac{\delta t}{2}\left(V_2 + \frac{1}{\delta t^2}\right)
        & -i\frac{\delta t}{(2 \delta x)^2} & 0 & 0 & \cdots & 0 \\
        0 & \ddots & \ddots & \ddots & 0 & \cdots & 0 \\
        \vdots & & & & & & \vdots \\
        0 & \cdots & \cdots & 0 &
        -i\frac{\delta t}{(2 \delta x)^2}
        &
        1 + i\frac{\delta t}{2}\left(V_2 + \frac{1}{\delta t^2}\right)
        & -i\frac{\delta t}{(2 \delta x)^2}
    \end{bmatrix}
    \begin{bmatrix}
        \Psi_1(\delta t) \\
        \Psi_2(\delta t) \\
        \vdots \\
        \\
        \Psi_N(\delta t)
    \end{bmatrix}
    =
    \vec{\phi}_j(0)
    $}
    \label{eq:prop}
\end{gather}

where the matrix in equation \eqref{eq:prop} is the propagator matrix.

\addcontentsline{toc}{subsection}{Finite Square Potential Well}
\subsection*{Finite Square Potential Well}
A potential well is a region where the wave packet experiences a lower
amount of potential energy. A well can be modeled by defining the potential
energy function $V(x)$ to include a region with a lower value.

In computer implementation, the well must be continuous due to the Gibbs
phenomenon. A smooth square well that does not allow for the occurance of
Gibbs phenomenon can be achieved by using a higher power Gaussian function.
In my modeling, I set the width of the well to be smaller than the width of
the wave packet, which will be discussed in more detail within the
\nameref{sec:Results} section.

\addcontentsline{toc}{subsubsection}{Transmission Coefficient}
\subsubsection*{Transmission Coefficient}
As the wave propagates through the finite potential well, some of it is
reflected backward, and some of it is transmitted through to the other side
of the well.

Suppose a potential well has a width of $2a$ and is centered at $x = 0$,
where the potential $V(x)$ is equal to $-V_0$ for $-a \leq x \leq a$ and $0$
for $x > a$ and $x < -a$. We label the region corresponding to $x < -a$ as
region I, to $-a < x < a$ as region II, to $x > a$ as region III. During our
meetings we solved for the one-dimentional time-independent Schrödinger
equation for when $E > V(x)$ and $E < V(x)$, obtaining

\begin{align*}
    \Psi_\mathrm{I}(x) &= Ae^{ikx} + Be^{-ikx} \\
    \Psi_{\mathrm{II}}(x) &= C\sin(lx) + D\cos(lx) \\
    \Psi_{\mathrm{III}}(x) &= Fe^{ikx}
\end{align*}

where $l = \sqrt{2(E + V_0)}$, $k = \sqrt{2E}$, and the subscript of the
denotation of each function indicates over which region the function is
defined. The requirements to stitch the three functions into one function
which is defined for all $x$ follows the properties of a valid solution to
the Schrödinger equation:

We compute the transmission coefficient ($T$) by definition ($T^{-1} =
\frac{\left| A \right| ^{2}}{\left| F \right| ^{2}}$), substituting $F$ with
equivalent expression in terms of the energy of the wave ($E$) and the depth
of the potential well.

The requirements to stitch the three wave functions into one follows the
properties of a valid solution to the Schrödinger equation:

\begin{gather}
    Ae^{-ika} + Be^{ika} = -C\sin(la) + D\cos(la) \label{eq:stitch1} \\
    C\sin(la) + D\cos(la) = Fe^{ika} \label{eq:stitch2} \\
    ik(Ae^{-ika} - Be^{ika}) = l(C\cos(la) + D\sin(la)) \label{eq:stitch3} \\
    l(C\cos(la) - D\sin(la)) = ikFe^{ika} \label{eq:stitch4}
\end{gather}

Now we solve for $F$ in terms of $l$, $k$, $a$, and $A$. From equation
\eqref{eq:stitch2}, we solve

\begin{equation}
    C = \frac{Fe^{ika} - D\cos(la)}{\sin(la)} \label{eq:CinD}
\end{equation}

and substitute into equation \eqref{eq:stitch4} to solve for

\begin{gather*}
    Fe^{ika}
    \left(
    \frac{ik}{l}\sin(la) - \cos(la)) = -D(\cos^2(la) + \sin^2(la)
    \right)
    \\
    D = Fe^{ika}
    \left(
    \cos(la) - \frac{ik}{l}\sin(la)
    \right)
\end{gather*}

Then we substitute $D$ in terms of $F$ into equation \eqref{eq:CinD} and
simplify. Note $1 - \cos^2(la) = 1 - \frac{1}{2}(1 + \cos(2la)) = 1 -
\frac{1}{2}(1 + 1 - 2\sin^2(la)) = \sin^2(la)$.

\[
    \scalebox{1}{$
    C
    = \frac{Fe^{ika}(1 - \cos^2(la) +
    \frac{ik}{l}\cos(la)\sin(la))}{\sin(la)}
    = Fe^{ika}
    \left(
    \frac{1 - \cos^2(la)}{\sin(la)} + \frac{ik}{l}\cos(la)
    \right)
    $}
\]

\[
    C = Fe^{ika}
    \left(
    \sin(la) + \frac{ik}{l}\cos(la)
    \right)
\]

Note $\cos^2(la) - \sin^2(la) = \cos(2la)$ and $2\cos(la)\sin(la) =
\sin(2la)$. Substituting $D$ and $C$ in terms of $F$ into equation
\eqref{eq:stitch1}, we get

\begin{gather}
    \scalebox{0.9}{$
    Ae^{-ika}
    = Fe^{ika}
    \left(
    -\sin^2(la) - \frac{ik}{l}\cos(la)\sin(la) + \cos^2(la)
    - \frac{ik}{l}\sin(la)\cos(la)
    \right)
    - Be^{ika}
    \nonumber
    $}
    \\
    = Fe^{ika}
    \left(
    \cos(2la) - \frac{ik}{l}\sin(2la)
    \right)
    - Be^{ika}. \label{eq:AinFB}
\end{gather}

Now we substitute this result and $D$ and $C$ in terms of $F$ into equation
\eqref{eq:stitch3} and solve for $B$.

\begin{align*}
    \scalebox{0.8}{$
    Fe^{ika}
    \left(
    -\frac{ik}{l}\sin(2la) + \cos(2la)
    \right)
    - 2Be^{ika}
    $}
    &=
    \scalebox{0.8}{$
    \frac{l}{ik}Fe^{ika}
    \left(
    2\cos(la)\sin(la) + \frac{ik}{l}\cos^2(la) - \frac{ik}{l}\sin^2(la)
    \right)
    $}
    \\
    -2B &= F\sin(2la)
    \left(
    \frac{l}{ik} + \frac{ik}{l}
    \right)
    \\
    B &= i\frac{\sin(2la)}{2kl}(l^2 - k^2)F
\end{align*}

To solve for $F$ in terms of $A$ we substitute $B$, $C$, and $D$ into
equation \eqref{eq:AinFB}.

\begin{gather*}
    Ae^{-ika}
    = Fe^{ika}
    \left(
    \cos(2la) - \frac{ik}{l}\sin(2la)
    - i\frac{\sin(2la)}{2kl}(l^2 - k^2)
    \right)
    \\
    Ae^{-2ika}
    = Fe^{ika}
    \left(
    \cos(2la) -
    \left(
    \frac{ik}{l} + \frac{i(l^2 - k^2)}{2lk}
    \right)
    \sin(2la)
    \right)
    \\
    F = \frac{e^{-2ika}A}{\cos(2la) - i
    \left(
    \frac{k^2 + l^2}{2kl}
    \right)
    \sin(2la)}
\end{gather*}

Finally, we compute $T^{-1}$ and rewrite it in terms of $E$ and $V_0$.

\begin{gather*}
    T^{-1} = \frac{\left| A \right| ^{2}}{\left| F \right| ^{2}}
    = \frac{\left| A \right| ^{2}}{
        \frac{\left| e^{-2ika} \right| ^2 \left| A \right | ^2}{
                \left|
                \cos^2(2la) + i \left( \frac{k^2 + l^2}{2kl} \right)
                \sin(2la)
                \right| ^2
            }
        }
    = \cos^2(2la) + \left( \frac{(k^2 + l^2)^2}{4k^2l^2} \right) \sin^2(2la)
    \\
    = 1 - \sin^2(2la) + \left( \frac{(k^2 + l^2)^2}{4k^2l^2}
    \right) \sin^2(2la)
    = 1 + \left( \frac{k^4 - 2k^2l^2 + l^4}{4k^2l^2} \right) \sin^2(2la)
    \\
    = 1 + \left( \frac{4E^2 - 8E(E + V_0) + 4(E^2 + 2EV_0 + V_0^2)}
    {16E(E + V_0)} \right) \sin^2(2a\sqrt{2(E + V_0)}
\end{gather*}

\begin{center}
    \boxed{
        T = \frac{1}{1 + \left( \frac{V_0}{4E(E + V_0)} \right)
        \sin^2(2a\sqrt{2(E + V_0)}}
    }
\end{center}

Note that T oscillates with changes in the width of the well and the energy
of the wave.

\addcontentsline{toc}{subsection}{Expected Values over Time}
\subsection*{Expected Values over Time}
\label{sec:expected}

In the \nameref{subsec:verify} subsection, the numerically calculations of
the expected values of certain physical observables of the wave packet
through time will be presented to gain confidence in correctness of my
programs. In this section, we will study the theoretic evolution of those
values, and the potential energy functions of interest are determined by
systems which were modeled computationally.

Both systems are closed, so energy is conserved and the value of the
Hamiltonian does not change.

\addcontentsline{toc}{subsubsection}{Constant Potential Energy}
\subsubsection*{Constant Potential Energy}

From equation \eqref{eq:EnoV}, we can deduce that $p \approx k$. As the wave
packet approaches a wall, the expected momentum will decrease as some of the
wave packet will have been reflected backward, while the rest of the wave
packet is still moving forward. The expected momentum will continue to
decrease until the wave packet has been fully reflected, at which point $p
\approx -k$.

Recall that a wavepacket is a superposition of eigenstates, and that it has
multiple, distinct $k$ values. One may calculate the distribution of the $k$
values from a Fourier transform, and would notice that it corresponds to the
Gaussian envelope of the wave packet. The components of the wave packet with
larger corresponding wave numbers will have larger momentum than those with
smaller corresponding wave numbers, so at every reflection, change in
momentum after the inflection point is less drastic.

The expected momentum is proportional to the change in the expected position
(in our case, it exactly describes the motion of the wave packet because we
assume that mass is one), so the expected $x$ would increase to a certain
point linearly, then decrease linearly to a certain point, and so on.
Because of the range of wave numbers discussed above, the change in $x$ of
the wave packet becomes larger, and the maxiumum expected $x$ for each
period is less than that for the previous period.

\addcontentsline{toc}{subsubsection}{Finite Square Potential Well}
\subsubsection*{Finite Square Potential Well}

Just as in a system with a constant potential function, the momuntum of the
wave packet is initially approximately $k$ in a system with a finite
potential well. However, once the wave packet passes the region with lower
potential, part of it is transmitted as a wave packet whose momentum has the
same sign as the momentum of the original wave packet, and the rest
reflected as another wave packet whose momentum has the opposite sign as the
momentum of the original wave packet. Otherwise, it is negative. The larger
the probability of the particle that is localized by the wave packet being
past the potential well after transmission, the larger the expected momentum
will be. If more of the wave packet is transmitted, the momentum between the
time of finishing transmission and reflection off the boundaries of the
infinite square well is positive.

Of course, the behavior of the momentum at reflection off the boundaries of
the box is consistent with that within a box with constant potential, and
the consequences of the wave packet having a range of wave numbers holds in
this system. Again, the change of the expected position is proportional to
the momentum.

\paragraph{Effects of Changing Well Width and Energy}  We know that the
behavior of expected momentum and position in a finite square well are
contingent on the transmission of the initial wave packet(s) and that the
transmission changes with the width of the well and the energy of the wave
packet. Thus the behavior of the those expected values change with the width
of the well and the energy of the wave packet. Changing the value of either
with the other held constant oscillates the value of the transmission
coefficient, and the changes in the evolution of expected momentum and
expected position with respect to the width of the well and the energy of
the wave packet follow from the descriptions detailed above.
