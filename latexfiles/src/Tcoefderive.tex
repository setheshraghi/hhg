\documentclass[letterpaper,12pt]{article}
\usepackage[margin=1in]{geometry}
\usepackage{fancyhdr}
\usepackage{amsfonts,amsmath,amsthm,amssymb,mathtools,empheq,array}
\usepackage{pgfplots,float,parskip}
\usepackage{pythonhighlight}

\graphicspath{{../assets/}}
\pagestyle{fancy}
\fancyhf{}
\lhead{Seth Eshraghi, \today}
\rfoot{Page \thepage}

\begin{document}

    \section*{\sloppy Transmission Coeffecient of a Finite Potential Well
    Derivation}

    We compute the transmission coefficient ($T$) by definition ($T^{-1} =
    \frac{\left| A \right| ^{2}}{\left| F \right| ^{2}}$), substituting $F$ with
    equivalent expression in terms of the energy of the wave ($E$) and the
    depth of the potential well. We choose units such that $\hbar = 1$ and
    asuume $m = 1$.

    Suppose a potential well has a width of $2a$ and is centered at $x = 0$,
    where the potential $V(x)$ is equal to $-V_0$ for $-a \leq x \leq a$ and $0$
    for $x > a$ and $x < -a$. We label the region corresponding to $x < -a$ as
    region I, to $-a < x < a$ as region II, to $x > a$ as region III. During our
    meetings we solved for the one-dimentional time-independent Schrödinger
    equation for when $E > V(x)$ and $E < V(x)$, obtaining

    \begin{align*}
        \Psi_\mathrm{I}(x) &= Ae^{ikx} + Be^{-ikx} \\
        \Psi_{\mathrm{II}}(x) &= C\sin(lx) + D\cos(lx) \\
        \Psi_{\mathrm{III}}(x) &= Fe^{ikx}
    \end{align*}

    where $l = \sqrt{2(E + V_0)}$, $k = \sqrt{2E}$ and the subscript of the
    denotation of each function indicates over which region the function is
    defined. The requirements to stitch the three functions into one function
    follows the properties of a valid solution to the Schrödinger equation:

    \begin{gather}
        Ae^{-ika} + Be^{ika} = -C\sin(la) + D\cos(la) \label{stitch1} \\
        C\sin(la) + D\cos(la) = Fe^{ika} \label{stitch2} \\
        ik(Ae^{-ika} - Be^{ika}) = l(C\cos(la) + D\sin(la)) \label{stitch3} \\
        l(C\cos(la) - D\sin(la)) = ikFe^{ika} \label{stitch4}
    \end{gather}

    Now we solve for $F$ in terms of $l$, $k$, $a$, and $A$. From
    equation \eqref{stitch2}, we solve

    \begin{equation}
        C = \frac{Fe^{ika} - D\cos(la)}{\sin(la)} \label{CinD}
    \end{equation}

    and substitute into equation \eqref{stitch4} to solve for

    \begin{gather*}
        Fe^{ika}
        \left(
        \frac{ik}{l}\sin(la) - \cos(la)) = -D(\cos^2(la) + \sin^2(la)
        \right)
        \\
        D = Fe^{ika}
        \left(
        \cos(la) - \frac{ik}{l}\sin(la)
        \right)
    \end{gather*}

    Then we substitute $D$ in terms of $F$ into equation \eqref{CinD} and
    simplify. Note $1 - \cos^2(la) = 1 - 1 - \frac{1}{2}(1 + \cos(2la)) = 1 -
    \frac{1}{2}(1 + 1 - 2\sin^2(la)) = \sin^2(la)$.

    \[
        C
        = \frac{Fe^{ika}(1 - \cos^2(la) +
        \frac{ik}{l}\cos(la)\sin(la))}{\sin(la)}
        = Fe^{ika}
        \left(
        \frac{1 - \cos^2(la)}{\sin(la)} + \frac{ik}{l}\cos(la)
        \right)
    \]

    \[
        C = Fe^{ika}
        \left(
        \sin(la) + \frac{ik}{l}\cos(la)
        \right)
    \]

    Note $\cos^2(la) - \sin^2(la) = \cos(2la)$ and $2\cos(la)\sin(la) =
    \sin(2la)$. Substituting $D$ and $C$ in terms of $F$ into equation
    \eqref{stitch1}, we get

    \begin{gather}
        Ae^{-ika}
        = Fe^{ika}
        \left(
        -\sin^2(la) - \frac{ik}{l}\cos(la)\sin(la) + \cos^2(la)
        - \frac{ik}{l}\sin(la)\cos(la)
        \right)
        - Be^{ika}
        \nonumber
        \\
        = Fe^{ika}
        \left(
        \cos(2la) - \frac{ik}{l}\sin(2la)
        \right)
        - Be^{ika}. \label{AinFB}
    \end{gather}

    Now we substitute this result and $D$ and $C$ in terms of $F$ into equation
    \eqref{stitch3}, and solve for $B$.

    \begin{align*}
        Fe^{ika}
        \left(
        -\frac{ik}{l}\sin(2la) + \cos(2la)
        \right)
        - 2Be^{ika}
        &=
        \frac{l}{ik}Fe^{ika}
        \left(
        2\cos(la)\sin(la) + \frac{ik}{l}\cos^2(la) - \frac{ik}{l}\sin^2(la)
        \right)
        \\
        -2B &= F\sin(2la)
        \left(
        \frac{l}{ik} + \frac{ik}{l}
        \right)
        \\
        B &= i\frac{\sin(2la)}{2kl}(l^2 - k^2)F
    \end{align*}

    To solve for $F$ in terms of $A$ we substitute $B$, $C$, and $D$ into
    equation \eqref{AinFB}.

    \begin{gather*}
        Ae^{-ika}
        = Fe^{ika}
        \left(
        \cos(2la) - \frac{ik}{l}\sin(2la)
        - i\frac{\sin(2la)}{2kl}(l^2 - k^2)
        \right)
        \\
        Ae^{-2ika}
        = Fe^{ika}
        \left(
        \cos(2la) -
        \left(
        \frac{ik}{l} + \frac{i(l^2 - k^2)}{2lk}
        \right)
        \sin(2la)
        \right)
        \\
        F = \frac{e^{-2ika}A}{\cos(2la) - i
        \left(
        \frac{k^2 + l^2}{2kl}
        \right)
        \sin(2la)}
    \end{gather*}

    \pagebreak

    Finally, we compute $T^{-1}$ and rewrite it in terms of $E$ and $V_0$.

    \begin{gather*}
        T^{-1} = \frac{\left| A \right| ^{2}}{\left| F \right| ^{2}}
        = \frac{\left| A \right| ^{2}}{
            \frac{\left| e^{-2ika} \right| ^2 \left| A \right | ^2}{
                    \left|
                    \cos^2(2la) + i \left( \frac{k^2 + l^2}{2kl} \right)
                    \sin(2la)
                    \right| ^2
                }
            }
        = \cos^2(2la) + \left( \frac{(k^2 + l^2)^2}{4k^2l^2} \right) \sin^2(2la)
        \\
        = 1 - \sin^2(2la) + \left( \frac{(k^2 + l^2)^2}{4k^2l^2}
        \right) \sin^2(2la)
        = 1 + \left( \frac{k^4 - 2k^2l^2 + l^4}{4k^2l^2} \right) \sin^2(2la)
        \\
        = 1 + \left( \frac{4E^2 - 8E(E + V_0) + 4(E^2 + 2EV_0 + V_0^2)}
        {16E(E + V_0)} \right) \sin^2(2a\sqrt{2(E + V_0)}
    \end{gather*}

    \begin{center}
        \boxed{
            T = \frac{1}{1 + \left( \frac{V_0}{4E(E + V_0)} \right)
            \sin^2(2a\sqrt{2(E + V_0)}}
        }
    \end{center}
\end{document}
